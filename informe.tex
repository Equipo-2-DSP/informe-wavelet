\documentclass[]{IEEEtran}
\usepackage[utf8]{inputenc}
\usepackage[spanish,es-tabla]{babel}
\usepackage{amsmath}
\usepackage{amsfonts}
\usepackage{amssymb}
\usepackage{graphicx}

\title{}
\author{}
\date{}

\newcommand{\wavelet}{\hspace{.002cm}\textbf{Wavelet}\hspace{.002cm} }

\begin{document}
    \maketitle

    \begin{abstract}

    \end{abstract}

    \section{Introducción}

    \section{Motivación}


    \section{Marco Teoríco}


    \section{Aplicaciones}

    Las aplicaciones que fueron seleccionadas son:

    \begin{enumerate}
        \item Filtrado de ruido.
        \item Compresión de imagenes.
        \item Detección de maximos en tiempo y frecuencia.
    \end{enumerate}


    \subsection{Filtrado de ruido}

    Es muy habitual que al enviar información ya sea de forma analogica o digital al llegar a destino 
    esta contenga ruido, la transformada de \wavelet permite eliminar de forma sencilla el ruido.
    Esto se logra descomponiendo la señal mediante la transformada de \wavelet en multiples niveles, 
    la cantidad maxima de niveles se ve definida por la \wavelet madre elegida. 
    Luego se procede a eliminar la componentes que posean una magnitud menor a un umbral, 
    para definir el umbral se pueden utilizar varios ecuaciones aunque la mayoria se basan 
    en calcular el desvio estandar de la señal.

    \subsection{Compresión de imagenes}

    

    \subsection{Detección de maximos en tiempo y frecuencia}

    Esta aplicaciones es muy util en la detección de sismos ya que permite obtener su
    magnitud y en que instante de tiempo sucecidio de forma muy sencilla. 
    Ya que solo se debe buscar un maximo de amplitud en la transformda \wavelet de la señal, 
    esto conyeva a buscar un maximo en 2 dimensiones. 

    \section{E}


\end{document}